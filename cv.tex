\documentclass[11pt]{article}

\newcounter{papers}

\usepackage[style=numeric,sorting=none]{biblatex}%<- specify style
\renewbibmacro{in:}{}
\addbibresource{cv.bib}%<- specify bib file

\usepackage[margin=0.75in]{geometry}
\makeatletter
\def\vhrulefill#1{\leavevmode\leaders\hrule\@height#1\hfill \kern\z@}
\makeatother
\begin{document}
%\nobibliography{refs}
%\bibliographystyle{unsrt}	
\begin{center}
  \textbf{DEBORAH FERGUSON}\\
  School of Physics, Center for Gravitational Physics\\
  University of Texas at Austin\\
\end{center}  

\begin{flushleft}

  Email: deborah.ferguson@austin.utexas.edu\\
  Webpage: deborahferguson.info\\
  \vspace{8px}
  
  \textbf{Education:}\\
  Ph.D., Physics, Georgia Institute of Techcnology, 2020\\
  M.S., Physics, Georgia Institute of Techcnology, 2017\\
  B.S., Physics, University of Kentucky, 2016, \textit{summa cum laude} \\ Mathematics and Computer Science Minors\\

\vspace{8px}

  \textbf{Research Positions:}\\
  \vspace{4px}
 
  \textit{Postdoctoral Fellow} \hfill \textit{2021-Present} \\
  University of Texas at Austin\\
  Research focus: Optimizing the use of numerical relativity with next generation\\ gravitational wave detectors\\
  Advisor: Deirdre Shoemaker\\
  
 \vspace{8px}  
  
  \textit{Graduate Research Assistant} \hfill \textit{2016-2020}\\
  Georgia Institute of Technology\\
  Dissertation title: Strong-Field Numerical Relativity in the Era of Gravitational\\ Wave Astronomy\\
  Advisor: Deirdre Shoemaker\\
  
  \vspace{8px}
	
  \textit{Undergraduate Research Assistant} \hfill \textit{2013-2017}\\
  University of Kentucky\\
  Research focus: Milky Way tomography with K and M dwarf Stars: the vertical\\ structure of the galactic disk\\ 
  Advisor: Susan Gardner\\
  \vspace{2px}
  Research focus: Using resonant frequencies and overtones to calculate the\\ sound velocity of TIPS Pentacene\\
  Advisor: Joseph Brill\\
  \vspace{2px}
  Research focus: Developing a mobile physics engine modeling two dimensional\\ elastic and inelastic collisions for the educational demonstration of physics phenomena\\
  Advisor: Jerzy Jaromczyk\\
  
  \vspace{8px}
	
  \textbf{Honors and Awards:}\\
  2022 - Visualizing Science Competition (University of Texas at Austin)\\
  2021 - Third Place in Visualizing Science Competition (University of Texas at Austin)\\
  2020 - Larry S. O'Hara Graduate Student Fellowship (Georgia Institute of Technology)\\
  2020 - Amelio Award for Research Excellence (Georgia Institute of Technology)\\
  2019 - Lindau Nobel Laureate Meeting (Oak Ridge and Lindau Committee)\\
  2016 - Georgia Tech Institute Fellowship (Georgia Institute of Technology)\\
  2016 - Outstanding Senior in Physics (University of Kentucky)\\
  2015 - Oswald Award Honorable Mention (University of Kentucky)\\
  2015 - First Place in University of Kentucky ACM Chapter Algorithmic Programming Competition\\ 
  \hspace{29px} (University of Kentucky)\\
  2015 - Summer Research Grant (University of Kentucky)\\
  2015 - Outstanding Junior in Physics (University of Kentucky)\\
  2014 - 42nd in United States in IEEEXtreme 24-Hour Programming Competition (IEEE)\\
  2013 - Singletary Scholarship (University of Kentucky)\\
  2013 - National Merit Finalist (National Merit Scholarship Foundation)\\

\vspace{8px}

\textbf{Invited Talks:}
  \begin{enumerate}
  \item ``Numerical Relativity with an Eye Toward Next Generation Gravitational Wave Detectors'', Seminar at Max Planck Institute, Albert Einstein Institute, Potsdam Germany, October 12, 2022 
  \item ``Waveforms for Gravitational Wave Astronomy'', Lecture at Petnica Summer Institute, Valjevo, Serbia, August 9, 2022
  \item ``Gravitational Wave Data Analysis'', North American Einstein Toolkit Workshop, July 29, 2021
  \item ``Assessing the Preparedness of Numerical Relativity for Current and Future Gravitational Wave Detectors'', Seminar at Gravity Group, University of Mississippi, Oxford, MS, March 18, 2021
  \item ``Assessing the Preparedness of Numerical Relativity for Current and Future Gravitational Wave Detectors'', Seminar at Astrophysics/Gravity/Cosmology Group, University of Illinois Urbana-Champaign, Champaign, IL, November 11, 2020
  \end{enumerate}

  \textbf{Conference Presentations:}
  \begin{enumerate}
  \item D. Ferguson, ``Optimizing Numerical Relativity Placement using Neural Networks'', 14th International LISA Symposium, Virtual, July 2022
  \item D.Ferguson, ``Predicting Match to Optimize Numerical Relativity Template Placement'', American Physical Society April Meeting, New York City NY, April 2022
  \item D.Ferguson, D.Shoemaker, ``Impact of Secondary Spin in Black Hole Binaries with Increasing Mass Ratio'', 14th Edoardo Amaldi Conference on Gravitational Waves, Virtual, July 2021
  \item D.Ferguson,  D.Shoemaker, ``Impact of Secondary Spin in Black Hole Binaries with Increasing Mass Ratio'',  American Physical Society April Meeting, Virtual, April 2021
  \item D. Ferguson, K. Jani, P. Laguna, D. Shoemaker, ``Assessing the Readiness of Numerical Relativity for LISA'', 13th International LISA Symposium, Virtual, September 2020
  \item D. Ferguson, K. Jani, D. Shoemaker, ``Assessing the Readiness of Numerical Relativity for Future Gravitational Wave Detections'', American Physical Society April Meeting, Virtual, April 2020
  \item D. Ferguson, K. Jani, D. Shoemaker, ``Assessing the Readiness of Numerical Relativity for Future Gravitational Wave Detections'', 235th Meeting of the American Astronomical Society, Honolulu HI, January 2020
  \item D. Ferguson, ``Using Machine Learning to Optimize Placement of Numerical Relativity Simulations'', Machine Learning in Science and Engineering Conference, Atlanta GA, June 2019
  \item D. Ferguson, K. Jani, D. Shoemaker, poster, ``Numerical Relativity Errors in LISA'', LISA Waveform Working Group Meeting, Golm Germany, May 2019
  \item D. Ferguson, et al., ``Revealing the Final Black Hole from Signal at Maximum Amplitude'', American Physical Society April Meeting, Columbus OH, April 2018
  \item D. Ferguson, et al., ``Apparent Horizon Dynamics of Binary Black Hole Systems'', Georgia Regional Astronomers Meeting, Athens GA, October 2017
  \item C. Powell, D. Ferguson, ``A Mobile Physics Engine Modeling Two Dimensional Elastic and Inelastic Collisions for the Educational Demonstration of Physics Phenomena''. National Conference on Undergraduate Research, Lexington KY, April 2014
  \end{enumerate} 

  \textbf{Short Author Publications:}
  \begin{enumerate}  
  	\item \fullcite{Ferguson:2022qkz}
	\item \fullcite{Ferguson:2020xnm}
    \item \fullcite{Evans:2020lbq}
    \item \fullcite{Ferguson:2019slp}
    \item \fullcite{Heal:2017abq}
    \item \fullcite{2017ApJ...843..141F}
    \setcounter{papers}{\value{enumi}}
  \end{enumerate}
  
  \textbf{Collaboration Publications with Direct Involvement:}  
  \begin{enumerate}
  \setcounter{enumi}{\value{papers}}
  \item \fullcite{LIGOScientific:2021sio}
  \item \fullcite{LIGOScientific:2020tif}	
  \setcounter{papers}{\value{enumi}}
  \end{enumerate}
  
  \textbf{Collaboration Publications:}
  \begin{enumerate}
  \setcounter{enumi}{\value{papers}}
  \item \fullcite{LIGOScientific:2022enz}
  \item \fullcite{LIGOScientific:2022lsr}
  \item \fullcite{LIGOScientific:2022jpr}
  \item \fullcite{LIGOScientific:2022myk}  
  \item \fullcite{LIGOScientific:2022cgf}
  \item \fullcite{LIGOScientific:2022pjk}
  \item \fullcite{LIGOScientific:2021quq}
  \item \fullcite{LIGOScientific:2021jlr}
  \item \fullcite{LIGOScientific:2021inr}
  \item \fullcite{LIGOScientific:2021hvc}
  \item \fullcite{LIGOScientific:2021psn}
  \item \fullcite{LIGOScientific:2021iyk}
  \item \fullcite{LIGOScientific:2021djp}
  \item \fullcite{LIGOScientific:2021aug}
  \item \fullcite{LIGOScientific:2021oez}
  \item \fullcite{LIGOScientific:2021job}
  \item \fullcite{LIGOScientific:2021ozr}
  \item \fullcite{LIGOScientific:2021usb}
  \item \fullcite{KAGRA:2021bhs}
  \item \fullcite{KAGRA:2021tnv}
  \item \fullcite{KAGRA:2021una}
  \item \fullcite{LIGOScientific:2021qlt}
  \item \fullcite{LIGOScientific:2021odm}
  \item \fullcite{LIGOScientific:2021mwx}
  \item \fullcite{Abbott:2021iab}
  \item \fullcite{LIGOScientific:2021yby}
  \item \fullcite{KAGRA:2021mth}
  \item \fullcite{LIGOScientific:2021nrg}
  \item \fullcite{KAGRA:2021kbb}
  \item \fullcite{Abbott:2020ilb}
  \item \fullcite{Abbott:2020mev}
  \item \fullcite{LIGOScientific:2020lst}
  \item \fullcite{Abbott:2020gyp}
  \item \fullcite{LIGOScientific:2020ibl}  
  \item \fullcite{Abbott:2020mjq}
  \item \fullcite{Abbott:2020tfl}
  \item \fullcite{Abbott:2020khf}
  \item \fullcite{LIGOScientific:2020stg}
  \item \fullcite{Abbott:2019ebz}
  \item \fullcite{Salemi:2019ovz}
  \end{enumerate}
  
  \textbf{Teaching:}\\
  Spring 2018 - Head TA for Introductory Physics I\\
  Fall 2017 - TA for Honors Introductory Physics II\\
  Summer 2017 - TA for Online Introductory Physics I\\
  Spring 2017 - TA for Introductory Physics II\\
  Fall 2017 - TA for Introductory Physics I\\
 
  \vspace{8px}   
 
  \textbf{Leadership and Outreach:}\\
  2022 - Talk at Lakeway Men's Breakfast Club\\
  2021 - Generated visualizations for LVC announcement of Black-Hole Neutron-Star mergers\\
  2020 - Generated visualizations for GW190521 announcement\\
  2019 - Atlanta Science Tavern Talk: Black holes wave back: observing the most violent events in the cosmos\\
  2018 - Served on Undergraduate Research Panel at Society of Women in Physics Conference\\
  2017-Present - Vertically Integrated Projects Mentor\\
  2017-2019 - Graduate Association of Physics Mentor\\
  2013-2016 - Presented at University of Kentucky Engineering Fair\\

  \vspace{8px}

  \textbf{Professional Service: }\\
  Co-organizer of UT Austin Theory Group Seminar\\
  Referee for Physical Review Letters, Physical Review D,  and General Relativity and Gravitation\\
  Scribe for NASA Review Panel\\
  
\end{flushleft}
\end{document}
